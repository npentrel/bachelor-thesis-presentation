%\begin{omgroup}{Lengths - Introduction to Measurements}
\begin{module}[id=lenght]
\begin{frame}
  \frametitle{Length}
  - basic measurements\\
  \begin{columns}
    \begin{column}{5.8cm}
      \begin{itemize}
      \item
      \begin{definition}
        \defi{length} is a geometric measurement for the dimension of objects such as:
        \begin{itemize}
        \item \defi{triangles},
        \item \defi{squares},
        \item \defi{linea}, \ldots
        \end{itemize}
      \end{definition}
    \end{itemize}
    \end{column}
    \begin{column}{5.5cm}
      \mhcgraphics[width=5.5cm]{measurements/lengthsquare.png}
      %source: http://www.themathpage.com/areal/App_IMG/179.gif
    \end{column}
  \end{columns}
\begin{itemize}
\item Different units are used around the world, we use the metric system
  	\begin{itemize}
  	\item SI (International System of Units) with the base unit meter
  	\item 	\begin{definition}
		  		\defi{meter} the length of the path travelled by light in vacuum during a time interval of 1/299,792,458 of a second.
			\end{definition}
  	\item Non-SI (the Centimeter–gram–second system of units) with the basic unit centimetre
  	\item 	\begin{definition}
			  \defi{centimeter} 1/100 of a meter.
			\end{definition}
  	%source: http://en.wikipedia.org/wiki/Unit_of_length
	\end{itemize}
\end{itemize}
\end{frame}
\end{module}
%\end{omgroup}
%%% Local Variables:
%%% mode: latex
%%% TeX-master: "all"
%%% End:

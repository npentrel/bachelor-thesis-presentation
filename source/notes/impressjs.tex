\begin{frame}
  \frametitle{impress.js}
  \begin{module}[id=impressjs]\\
  The open-source presentation framework \textbf{impress.js} was created by Bartek Szopka. \\
  The presentation data is entered in the HTML file; each slide in its own div with \texttt{data-x}, \texttt{data-y}, and \texttt{data-z} attributes to change the position of the slide.\\

\lstset{ %
  backgroundcolor=\color{white},   % choose the background color; you must add \usepackage{color} or \usepackage{xcolor}
  basicstyle=\footnotesize,        % the size of the fonts that are used for the code
  breakatwhitespace=false,         % sets if automatic breaks should only happen at whitespace
  breaklines=true,                 % sets automatic line breaking
  captionpos=b,                    % sets the caption-position to bottom
  commentstyle=\color{mygreen},    % comment style
  deletekeywords={...},            % if you want to delete keywords from the given language
  escapeinside={\%*}{*)},          % if you want to add LaTeX within your code
  extendedchars=true,              % lets you use non-ASCII characters; for 8-bits encodings only, does not work with UTF-8
  frame=single,                    % adds a frame around the code
  keepspaces=true,                 % keeps spaces in text, useful for keeping indentation of code (possibly needs columns=flexible)
  keywordstyle=\color{blue},       % keyword style
  language=Octave,                 % the language of the code
  otherkeywords={*,...},            % if you want to add more keywords to the set
  numbers=left,                    % where to put the line-numbers; possible values are (none, left, right)
  numbersep=5pt,                   % how far the line-numbers are from the code
  numberstyle=\tiny\color{mygray}, % the style that is used for the line-numbers
  rulecolor=\color{black},         % if not set, the frame-color may be changed on line-breaks within not-black text (e.g. comments (green here))
  showspaces=false,                % show spaces everywhere adding particular underscores; it overrides 'showstringspaces'
  showstringspaces=false,          % underline spaces within strings only
  showtabs=false,                  % show tabs within strings adding particular underscores
  stepnumber=2,                    % the step between two line-numbers. If it's 1, each line will be numbered
  stringstyle=\color{mymauve},     % string literal style
  tabsize=2,                       % sets default tabsize to 2 spaces
  title=\lstname                   % show the filename of files included with \lstinputlisting; also try caption instead of title
}

\begin{lstlisting}[language=HTML]
<div class="step" data-x="1000" data-y="0">
  Slide Content
</div>
\end{lstlisting}

\noindent
Apart from these basic attributes, impress.js offers the \texttt{data-scale} attribute to scale content, making slides appear bigger or smaller, and the \texttt{data-rotate} attribute which rotates slides in 3 dimensions using \texttt{data-rotate-x}, \texttt{data-rotate-y}, and \texttt{data-rotate-z}. This allows the presentation to go into the third dimension.\\
\noindent
The opacity attribute allows us to discriminate between active and inactive slides which allows us to hide content so as not show too much information to a user at once. Adding an overview can be accomplished with the last part of the CSS snippet and the HTML snippet below that.

\begin{lstlisting}[language=HTML]
.step { opacity: 0.2; }
.step.active { opacity: 1; }
.step-overview .step { opacity: 1; cursor: pointer; }
\end{lstlisting}

\begin{lstlisting}[language=HTML]
<div id="overview" class="step" data-x="8000" data-y="1000"
  data-scale="10"> </div>
\end{lstlisting}
\noindent
To link back to a slide, slides are accessible via IDs. The slide with the ID \texttt{conclusion} will thus be accessible by appending \texttt{\#/conclusion} to the end of the URL.

  \end{module}
\end{frame}
%%% Local Variables:
%%% mode: latex
%%% TeX-master: "all"
%%% End:

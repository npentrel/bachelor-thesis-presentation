%%%%%%%%%%%%%%%%%%%%%%%%%%%%%%%%%%%%%%%%%%%%%%%%%%%%%%%%%%%%%%%%%%%%%%%%%%%%%%%%%%%%%%%%%%%%%
% Pythagorean Theorem
% Copyright (c) 2015 Naomi Pentrel, released under the Creative Commons Share-Alike License
%%%%%%%%%%%%%%%%%%%%%%%%%%%%%%%%%%%%%%%%%%%%%%%%%%%%%%%%%%%%%%%%%%%%%%%%%%%%%%%%%%%%%%%%%%%%%% 

\documentclass[slides,book,sectocframes]{mikoslides}
\input{localpaths}
%\usepackage{amsmath,amssymb}
\usepackage{multirow,paralist}
\usepackage{wasysym,marvosym,eurosym}
\usepackage{lststex,lstomdoc}
\usepackage{smglom}
\usepackage{ded,calbf,url,rotating}
\usepackage{wrapfig,colortbl,wasysym}
\usepackage[show]{ed}
\usepackage{listings}
\usepackage{color}
\usepackage{quoting}
\quotingsetup{vskip=50pt}

\definecolor{mygreen}{rgb}{0,0.6,0}
\definecolor{mygray}{rgb}{0.5,0.5,0.5}
\definecolor{mymauve}{rgb}{0.58,0,0.82}

\usetikzlibrary{shapes}
\usetikzlibrary{shadows}
\usetikzlibrary{patterns}
\usetikzlibrary{arrows}
\usetikzlibrary{backgrounds}
\usetikzlibrary{mmt}
\usetikzlibrary{docicon}
\usetikzlibrary{tikzmark}
\usetikzlibrary{circuits.logic}
\usetikzlibrary{circuits.logic.CDH}
\usetikzlibrary{shapes.gates.logic.US}
\usetikzlibrary{decorations,decorations.markings,decorations.text,decorations.pathmorphing}
\ifnotes\usepackage[bookmarks,linkcolor=black,citecolor=black,urlcolor=black,colorlinks,breaklinks,bookmarksopen,bookmarksnumbered]{hyperref}\fi
\baseURI[\MathHub{}]{https://tnt.kwarc.info/repos/stc}
}

\begin{document}

\begin{frame}
  \begin{ttitle}
    \red{Bachelor Thesis Presentation}\\
    \blue{The Combination of Spatial Narrative and Semantic Closeness to Derive Visualizations of Information Graphs}\\
    \today
  \end{ttitle}\vspace*{2cm}
  \begin{center}
    \textsc{Naomi Pentrel}\\[1ex]
    % School of Engineering \& Science\\
    Jacobs University Bremen\\
    \url{n.pentrel@jacobs-university.de}\\
  \end{center}
\end{frame}

\mhinputref{main/titlePage.tex}

\begin{omgroup}{Preliminaries}
\mhinputref{preliminaries/preliminaries.tex}
\end{omgroup} \newpage

\begin{omgroup}{Towards a More Effective Presentation Format}
\mhinputref{researchProject/researchProjectOverview.tex}
\end{omgroup} \newpage

\begin{omgroup}{Conclusion}
\mhinputref{conclusion/conclusion.tex}
\end{omgroup}\newpage

\end{document}
 
%%% Local Variables: 
%%% mode: LaTeX 
%%% TeX-master: t
%%% End:  
% LocalWords:  kohlhase newpage printindex stex maketitle inputref
% LocalWords:  gencs clearpage setcounter tocdepth tableofcontents clearpage
% LocalWords:  repcomp compwww searchdeccomp kwarc

\begin{frame}
  \frametitle{Information Packaging}
  \begin{module}[id=infoPackaging]

    Within the study of \textit{information structure}, one discriminates between \textit{familiar/ old} and \textit{unfamiliar/ new} information [HRP$^+$02].\\

    \begin{definition} \defiii {Familiar/} {old} {information} is shared by speaker and addressee, i.e. it is in the intersection of knowledge of student and teacher.\end{definition}
    \begin{definition} \defiii {New/} {unfamiliar} {information} is not in the shared knowledge base or the content commons [Tea06]. \end{definition}

  \begin{center}
  \textit{"My sister went to the circus the other day; \underline{she} said \underline{it} was brilliant."}\\
  \end{center}
\noindent
 The accessibility of information depends on the relative \textit{newness}, i.e. recency of mention, of this information.\\

    \begin{definition} \defii {course-new} {information} is the information that the speaker is passing on to the addressees/students which generally depends on \textit{course-old} information. \end{definition}
    \begin{definition} \defii {course-old} {information} is information that is easily remembered. \end{definition}
    \begin{definition} \defii {course-ancient} {information} is information that might be "too old" to be easily remembered. \end{definition}

  \end{module}
\end{frame}
%%% Local Variables:
%%% mode: latex
%%% TeX-master: "all"
%%% End:

%odule[path=preliminaries/impressjs]{impressjs}
%dule[path=preliminaries/narrativePath]{narrativePath}
%ule[path=preliminaries/status]{status}

\begin{frame}
  \frametitle{Levels}
  \begin{module}[id=levels]

When writing a document or course notes in \stex, the creator general writes a top-level file such as \textit{notes.tex} from where other \TeX\ files are included. These included files again include further \TeX\ files etc.. 

We already know that a dependent piece of information infers that it is \textit{course-ancient} information. Anything that is on the same level or in the sublevels thereof is considered to be \textit{course-old} information.

\mhcgraphics[width=0.70\textwidth]{researchProject/levels.png}

When the user enters an excursion a 90 degree rotation around the y-axis occurs and the user can now follow that story line.

\lstset{ %
  backgroundcolor=\color{white},   % choose the background color; you must add \usepackage{color} or \usepackage{xcolor}
  basicstyle=\footnotesize,        % the size of the fonts that are used for the code
  breakatwhitespace=false,         % sets if automatic breaks should only happen at whitespace
  breaklines=true,                 % sets automatic line breaking
  captionpos=b,                    % sets the caption-position to bottom
  commentstyle=\color{mygreen},    % comment style
  deletekeywords={...},            % if you want to delete keywords from the given language
  escapeinside={\%*}{*)},          % if you want to add LaTeX within your code
  extendedchars=true,              % lets you use non-ASCII characters; for 8-bits encodings only, does not work with UTF-8
  frame=single,                    % adds a frame around the code
  keepspaces=true,                 % keeps spaces in text, useful for keeping indentation of code (possibly needs columns=flexible)
  keywordstyle=\color{blue},       % keyword style
  language=Octave,                 % the language of the code
  otherkeywords={*,...},            % if you want to add more keywords to the set
  numbers=left,                    % where to put the line-numbers; possible values are (none, left, right)
  numbersep=5pt,                   % how far the line-numbers are from the code
  numberstyle=\tiny\color{mygray}, % the style that is used for the line-numbers
  rulecolor=\color{black},         % if not set, the frame-color may be changed on line-breaks within not-black text (e.g. comments (green here))
  showspaces=false,                % show spaces everywhere adding particular underscores; it overrides 'showstringspaces'
  showstringspaces=false,          % underline spaces within strings only
  showtabs=false,                  % show tabs within strings adding particular underscores
  stepnumber=2,                    % the step between two line-numbers. If it's 1, each line will be numbered
  stringstyle=\color{mymauve},     % string literal style
  tabsize=2,                       % sets default tabsize to 2 spaces
  title=\lstname                   % show the filename of files included with \lstinputlisting; also try caption instead of title
}

\begin{lstlisting}[language=HTML]
<div class="step slide" data-x="8750" data-y="0" data-z="0"
  data-rotate-y="0"> Slide </div>
<div class="step slide" data-x="8750" data-y="800" data-z="0" 
  data-rotate-y="0"> Dependency 1  </div>
<div class="step slide" data-x="8750" data-y="800" data-z="0" 
  data-rotate-y="90"> Dependency 1 Continuation 1 </div>
<div class="step slide" data-x="8750" data-y="800" data-z="1250" 
  data-rotate-y="90"> Dependency 1 Continuation 2 </div>
\end{lstlisting}

\mhcgraphics[width=0.65\textwidth]{researchProject/60degree}

  \end{module}
\end{frame}
%%% Local Variables:
%%% mode: latex
%%% TeX-master: "all"
%%% End:

%odule[path=preliminaries/impressjs]{impressjs}
%dule[path=preliminaries/narrativePath]{narrativePath}
%ule[path=preliminaries/status]{status}
%ule[path=preliminaries/primitives]{primitives}

\begin{frame}
  \frametitle{Relational Presentation}
  \begin{module}[id=relationalPresentation]

Combining the information about narrative paths, dependencies, and levels, we derived an ordered information graph. Let us examine the structure:

The goal of this presentation is to have all the information the Pythagorean theorem depends on easily accessible so that Annie can access \textit{course-ancient} information easily.
\begin{itemize}
\item very structured
\item easily usable
\item prezis would be more graphical - full of metaphors and meaningful visual connections.
\end{itemize}

  \mhcgraphics[width=0.42\textwidth]{researchProject/Storytelling}


Usage of closeness and movement. The following semantic movements exist:\\
\vspace{-12pt}
\begin{enumerate}[topsep=0pt,itemsep=-1ex,partopsep=1ex,parsep=1ex]
\item movement along the x-axis,
\item movement along the y-axis, and
\item rotations.
\end{enumerate}
\vspace{5pt}

  \end{module}
\end{frame}
%%% Local Variables:
%%% mode: latex
%%% TeX-master: "all"
%%% End:

%impressjs}
%spatialNarrative}
\begin{frame}
  \frametitle{Research Project}
  \begin{module}[id=researchProject]
     
RPresentation itself operates in five main steps:\\
\begin{enumerate}
\item Get user input for the locations and names of folders.
\item Parse \sTeX to retrieve the order of slides.
\item Extract dependencies from relational info.
\item Extract necessary XHTML or PNG parts for slides.
\item Create presentation. impress.js allows us to use spatial narrative.
\end{enumerate}

\mhcgraphics[width=0.70\textwidth]{researchProject/Architecture}

\end{module}
\end{frame}
\mhinputref{researchProject/narrativePaths.tex}
\mhinputref{researchProject/orderedInfoGraphs.tex}
\mhinputref{researchProject/orderedInf2.tex}
\mhinputref{researchProject/levels.tex}
\mhinputref{researchProject/relationalPresentation.tex}
%%% Local Variables:
%%% mode: latex
%%% TeX-master: "all"
%%% End:

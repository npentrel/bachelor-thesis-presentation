%Dep: impressjs
%Dep: spatialNarrative}
\begin{frame}
  \frametitle{Research Project}
  \begin{module}[id=researchProject]\\
\noindent
RPresentation itself operates in five main steps:
\begin{enumerate}
\item Get user input for the locations and names of folders.
\vspace{-4mm}
\item Parse \sTeX to retrieve the order of slides.
\vspace{-4mm}
\item Extract dependencies from relational info.
\vspace{-4mm}
\item Extract necessary XHTML or PNG parts for slides.
\vspace{-4mm}
\item Create presentation. impress.js allows us to use spatial narrative.
\end{enumerate}

\mhcgraphics[width=0.70\textwidth]{researchProject/Architecture}

\end{module}
\end{frame}
\mhinputref{researchProject/narrativePaths.tex}
\mhinputref{researchProject/orderedInfoGraphs.tex}
\mhinputref{researchProject/orderedInf2.tex}
\mhinputref{researchProject/levels.tex}
\mhinputref{researchProject/levelsCode.tex}
\mhinputref{researchProject/levels2.tex}
\mhinputref{researchProject/relationalPresentation.tex}
%%% Local Variables:
%%% mode: latex
%%% TeX-master: "all"
%%% End:
